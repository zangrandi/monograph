\appendix

\chapter{Código do modelo}

\section{Classificador}

Classe utilizada as ações referentes à classificação das publicações.

\begin{lstlisting}
	require 'csv'
	require 'libsvm'

	class Classificador
		attr_reader :problema, :parametro, :termos_excecao, :treinador, :modelo, :publicacoes

		def initialize
			@problema = Libsvm::Problem.new
			@parametro = Libsvm::SvmParameter.new
			@treinador = Treinador.new
			@publicacoes = []
			inicializa_parametros
		end

		def inicializa_parametros
			@parametro.c = 0.1
			@parametro.cache_size = 1
			@parametro.eps = 0.001
		end

		def carregar_publicacoes_de_treino caminho
			@treinador.carregar_publicacoes_de_treino caminho
		end

		def definir_modelo
			@problema.set_examples(@treinador.classes, @treinador.vetores_de_treino)
			@modelo = Libsvm::Model.train(@problema, @parametro)
		end

		def carregar_publicacoes caminho
			CSV.open(caminho) do |csv|
				csv.each do |linha|
					@publicacoes << Publicacao.new(linha, @treinador)
				end
			end
		end

		def testar_modelo caminho_saida
			definir_modelo
			CSV.open(caminho_saida, "w") do |csv|
				@publicacoes.each do |publicacao|
					linha_csv = publicacao.linha_csv
					classe = @modelo.predict(publicacao.vetor)
					linha_csv[6] = classe
					linha_csv[7] = publicacao.cidade
					linha_csv[8] = publicacao.estado
					csv << linha_csv
				end
			end
		end
	end
\end{lstlisting}

\section{Publicação}

Classe utilizada para extração das informações das publicações como: termos, vetor de características, cidade e estado.

\begin{lstlisting}
	class Publicacao
	  attr_reader :termos, :vetor, :linha_csv, :texto, :horario, :latitude, 
	  	:longitude, :cidade, :localizacao_no_perfil

		def initialize linha, treinador = nil
			@linha_csv = linha
			@id = @linha_csv[0]
			@horario = @linha_csv[1]
			@texto = @linha_csv[2]
			@latitude = @linha_csv[3]
			@longitude = @linha_csv[4]
			@localizacao_no_perfil = @linha_csv[5]
			@treinador = treinador
			@termos = @texto.tokenizar
			@cidade = extrair_cidade
		end

		def extrair_cidade
			cidade = ""
			@termos.each do |termo|
				if String::LISTA_DE_CIDADES.include?(termo)
					cidade = termo
				end
			end
			if cidade == "" && @localizacao_no_perfil
				@localizacao_no_perfil.tokenizar.each do |termo|
					if String::LISTA_DE_CIDADES.include?(termo)
						cidade = termo
					end
				end
			end
			cidade
		end

		def estado
			String::ESTADOS_E_CIDADES[@cidade]
		end

		def vetor
			if @treinador
				vetor = @treinador.dicionario_de_termos.map do |termo| 
					@termos.include?(termo) ? 1 : 0
				end
				Libsvm::Node.features(vetor)
			end
		end
	end
\end{lstlisting}

\section{Treinador}

Classe utilizada para a fase de treinamento do classificador e criação do dicionário de termos.

\begin{lstlisting}
	class Treinador
		attr_reader :publicacoes_de_treino, :publicacoes_de_teste, :dicionario_de_termos

		def initialize
			@publicacoes_de_treino = []
		end

		def carregar_publicacoes_de_treino caminho
			CSV.open(caminho) do |csv|
				csv.each do |linha|
					@publicacoes_de_treino << Publicacao.new(linha, self)
				end
			end
			criar_dicionario_de_termos
		end

		def criar_dicionario_de_termos
			@dicionario_de_termos = @publicacoes_de_treino.map do |publicacao| 
				publicacao.termos
			end.flatten.uniq.sort
		end

		def classes
			@publicacoes_de_treino.map { |publicacao| publicacao.linha_csv[6].to_i }
		end

		def vetores_de_treino
			@publicacoes_de_treino.map { |publicacao| publicacao.vetor }
		end
	end
\end{lstlisting}

\section{Twitter}

Classe utilizada para obter as publicações do Twitter.

\begin{lstlisting}
	require 'twitter'

	class Twitter
		def initialize
			@cliente = Twitter::REST::Client.new do |config|
			  config.consumer_key        = "DbT8fYWR2jq1TIXvxVtiZzFno"
			  config.consumer_secret     = "3mT9JcwgaWs5gTJqOihbGAlRplHTKQapJMNQtnAWufYMVyUmle"
			  config.access_token        = "42659961-4wDJCea7t1KVf1FMqUK5H2k2zgBaE26GpI5kjC3CK"
			  config.access_token_secret = "OjhVszJ3Dyivaz0gGhHUuKn5N7cYqE0jUinoaFiWK7FjY"
			end
		end

		def buscar data_limite, expressao, caminho_arquivo
			data_publicacao = Time.now
			id_maximo = nil

			while publication_date > data_limite
				publicacaoes = @cliente.search(expressao, lang: "pt", max_id: id_maximo).to_a
				data_publicacao = publicacaoes.last.created_at
				id_maximo = publicacoes.last[:id] - 1
				CSV.open(caminho_arquivo, "a+") do |csv|
					publicacoes.each do |publicacao| 
						csv << [publicacao[:id], publicacao.created_at, publicacao.text, publicacao.geo.latitude, publicacao.geo.longitude, publicacao.user.location]
					end
				end
			end
		end
	end
\end{lstlisting}

\section{String}

Sobrecarga da classe String nativa do Ruby para manipulação das cidades e tokenização das publicações.

\begin{lstlisting}
	require 'json'

	class String
		JSON_CIDADES = JSON.parse(IO.read("support/cidades_do_brasil.json"))
		ESTADOS_E_CIDADES = JSON_CIDADES.each_with_object({}) do |(k,v), h|
			h[k.downcase] = v.downcase
		end
		LISTA_DE_CIDADES = ESTADOS_E_CIDADES.keys.sort_by { |c| c.length }.reverse
		CIDADES_UNIDAS = LISTA_DE_CIDADES.map { |city| "(\\b#{city}\\b)" }.join("\|")
		CIDADES_EXPREG = Regexp.new "#{CIDADES_UNIDAS}|\\s"
		EXCECOES = [
			"a", "e", "o", "de", "do", "da", "dos", "das", 
			"os", "as", "se", "meu", "que", "teu", "ou", "eu",
			"é", "à", "um", "uma", "uns", "umas", "RT"
		]

		def tokenizar
			termos = self.downcase.split(CIDADES_EXPREG).map! { |termo| termo.gsub(/\p{^Alnum}\s/, '') }
			termos.delete_if do |termo| 
			  termo == "" || EXCECOES.include?(termo) || termo.include?("http") || 
			  termo.include?("@") || termo.include?("kk") || termo.include?("haha")
			end
		end
	end
\end{lstlisting}

\section{Modelo}

Programa principal que realiza as chamadas na ordem (a obtenção das publicações com a classe Twitter é executada separadamente).

\begin{lstlisting}
	["classificador", "publicacao", "string", "treinador"].each do |file|
		require "./lib/#{file}.rb"
	end

	classificador = Classificador.new
	classificador.carregar_publicacoes_de_treino "support/publicacoes_de_treino.csv"
	classificador.carregar_publicacoes "support/publicacoes.csv"
	classificador.testar_modelo "support/publicacoes_classificadas.csv"
\end{lstlisting}