 
\begin{resumo}

Detecção de eventos consiste no processo de identificação de padrões de mudança relevantes em um sistema. O modelo implementa a detecção de eventos com dados das publicações criadas por usuários no serviço de rede social Twitter, serviço que tem recebido grande atenção acadêmica por sua alta popularidade e característica de troca de informações em tempo real. O modelo utiliza, como exemplo de evento, manifestações que ocorrem no território brasileiro durante o mês de agosto de 2014. Primeiramente, são obtidas as publicações criadas no serviço através da sua interface para desenvolvedores. Posteriormente, as publicações são convertidas para o seu modelo de espaço vetorial, para que seus vetores de características sejam extraídos e que os mesmos sejam utilizados para a fase de classificação com máquina de vetores de suporte. A classificação consiste na divisão das publicações obtidas em duas classes: positivas e negativas. Sendo positivas as que dizem respeito à uma manifestação real, com local e horário, e negativas as que somente mencionam manifestação mas não dizem respeito à um acontecimento real. Após classificadas, as publicações são agrupadas e exibidas em gráficos no formato de série-temporal e mapas de marcadores, aonde é possível detectar, através da análise visual, tanto o horário dos eventos, através de picos que surgem nos gráficos, quanto a sua localização, através do seu aglomeramento em determinadas regiões do mapa.

\end{resumo}