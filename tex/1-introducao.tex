\chapter{Introdução}

A detecção de eventos é o processo de identificação de acontecimentos que fogem das regras normais de funcionamento de um sistema, ou de padrões de mudanças relevantes dentros dos mesmos. Segundo \citeonline{Wasson2006}, sistemas são:

\begin{quote}
	Um conjunto integrado de elementos interoperáveis, cada um com capacidade explicitamente especificadas e limitadas, que trabalham em sinergia para realizar o processamento de valor acrescentado para permitir que um usuário para satisfazer as necessidades operacionais orientadas para missões em um ambiente operacional prescrito com um resultado específico e probabilidade de sucesso.
\end{quote}

Sistemas incluem pessoas, peças, técnicas, processos, softwares, hardwares, etc, e podem ser reais ou virtuais (criados pelo homem). Para um sistema de monitoramento de temperatura de barras de ferro, por exemplo, um evento pode ser a temperatura da barra atingir um determinado valor. Para um sistema de monitoramento de tópicos em documentos de texto, um evento pode ser o repentino surgimento de documentos contendo um termo específico, o que identifica o surgimento de um novo tópico. Para cada implementação, o evento toma formas diferentes, de acordo com o interesse do agente implementador do sistema.

Como parte do sistema, existem os sensores, que são os componentes que produzem diretamente as entradas para a detecção. As saídas dos sensores são os dados vindo diretamente da análise do ambiente - temperatura, velocidade, som, documentos de texto, etc - e servem como entrada para a detecção de eventos \citeonline{Sakaki2010}. Sensores incluem de temperatura, pressão, altitude, geolocalização, mas também incluem qualquer outro componente que gera as entradas do detector, como relatórios, publicações de notícias ou serviços redes sociais. 

No caso de serviços redes sociais (SNS: Social Networking Services), um um usuário atua como um sensor, nesse caso sendo denominados como \textit{sensores sociais}. Os SNS são a principal forma com que pessoas se relacionam no ambiente virtual, neles são criadas, diaramente, milhões de publicações sobre os mais variados temas, o que tem chamado atenção acadêmica crescente nas áreas de mineração de dados, detecção de tópicos, detecção de eventos, etc. Dentre os SNS mais populares, está o Twitter, com sua base de 500 milhões de usuários e publicações com o limite de 140 caracteres, os usuários do serviço produzem uma imensa quantidade de informações. O limite permite com que as informações sejam trocadas de forma mais dinâmica, se aproximando ainda mais do tempo real dos acontecimentos do mundo. O dinanismo favorece a sua troca, produzindo redes que obtem e repassam as informações rapidamente. O usuário, nesse contexto, atua como um sensor dos eventos do mundo real, produzindo informações que podem ser utilizadas para detectá-los.

Exitem vários modelos matemáticos utilizados para a detecção de eventos, sendo os principais deles: métodos estatísticos, métodos probabilísticos e métodos de aprendizado de máquina. Os métodos estatísticos consistem, basicamente, na análise dos dados vindos dos sensores e na criação de um modelo estatístico para eles. Através da comparação entre os dados reais e o modelo estatístico, é possível indicar se em determinado momento, ocorre um evento fora do padrão ou não, através da diferença entre os valores esperados e reais. Os métodos probabísticos, consistem na criação de modelos probabilísticos para indicarem, a partir dos dados estatísticos, qual a probabilidade de ocorrer um evento em determinado momento. Enquanto os métodos de aprendizado de máquina analisam os padrões existentes em um conjunto de dados vindos dos sensores para determinar o funcionamento geral do sistema. São geralmente aplicados em dados esparsos e em sistemas que necessitam alta performance computacional.

Muitos estudos foram dedicados à detecção de eventos e ao Twitter. \citeonline{Becker2011} estudaram métodos de análise para detectar publicações do Twitter que estão relacionadas à eventos dos mundo real. \citeonline{Takahashi2011} e \citeonline{Sakaki2010} Outros trabalhos obtiveram sucesso ao utilizar sensores sociais para a detecção de eventos variados. \citeonline{Sakaki2010} desenvolve um algoritmo que detecta eventos de terremotos no Japão, superando o índice de acerto do orgão de meteorologia japonês Japan Meteorology Agency (JMA). Enquanto \citeonline{Takahashi2011} utiliza o Twitter como forma de detectar eventos de rinite alérgia no Japão.

\section{Metodologia}

O modelo implementado utiliza a detecção de eventos aplicada ao contexto do serviço de microblog \textit{Twitter}. O modelo utiliza sua interface de aplicação \textit{Search API} para realizar a busca por publicações criadas por seus usuários. As publicações são salvas localmente em um arquivo estruturado com dados como horário, geolocalização e cidade.

Muitas publicações, porém, mesmo contendo a palavra-chave buscada, podem não dizer respeito ao evento específico que se deseja detectar. É preciso, então, a aplicação de métodos que classifiquem essas publicações em relevantes para a análise ou não. Um dos métodos, é o método de aprendizado de máquina \textit{máquina de vetores de suporte} (SVM: \textit{Support Vector Machine}). O SVM irá analisar e identificar padrões a partir de um conjunto de publicações previamente classificadas e, então, gerar as classificações das novas publicações analisadas.

Para a detecção das publicações relevantes com SVM, é preciso que as mesmas sejam convertidas para o seu \textit{modelo de espaço vetorial}, que consiste no processo de transformação de dados para o seu vetor de características. O processo de conversão de documentos de texto consiste na tokenização, pré-processamento, a criação do dicionário de termos e do vetor de características.

Após a conversão, os documentos são inseridos como entrada para o SVM, que irá distribuir os seus vetores e, a partir da semelhança de suas características, determinar em qual classe cada dado se encontra. O processo de classifição com SVM é divido em três etapas: treino, teste e classificação. Na fase de treino, são selecionadas algumas publicações e classificadas manualmente para alimentar o conhecimento do SVM. Na fase de teste, uma outra gama de publicações é selecionada e o SVM executado para fazer o comparativo entre as suas saídas e as classificações manuais: assim determina-se a taxa de acerto do SVM. Com uma taxa de acerto aceitável, o SVM está apto à fase de classificação, na qual classifica-se o total de publicações obtida do Twitter.

Após a classificação das publicações, as mesmas tem as informações do horário e localização extraídas, para que os eventos sejam análisados visualmente através de gráficos de série-temporal e do mapa de marcadores de geolocalização.

\section{Objetivo}

Através deste trabalho, pretende-se desenvolvedor um modelo de detecção de eventos através do Twitter. O modelo utiliza o modelo de aprendizado de máquina SVM e a ajuda de gráficos de série-temporal interativos para detectar erupções anômalas de publicações, o que, na prática, determina a ocorrência de um evento. O modelo utiliza como exemplo de evento, manifestações que ocorrem no território brasileiro entre o período de 01 de agosto e 31 de agosto de 2014.

\section{Organização}

O trabalho está organizado da seguinte forma:

No Capítulo 2 são apresentadas as técnicas e conceitos relacionadps com detecção de eventos: a definição de eventos e sistemas, o que são sensores e qual a sua utilização para a detecção, a detecção de eventos em serviços de redes sociais, o que são serviços redes sociais, como o Twitter pode ser um meio para a aplicação da detecção e o que são sensores sociais, além de modelos matemáticos utilizados para a implementação da detecção de eventos e trabalhos relacionados.

No Capítulo 3 é demonstrado o modelo implementado: como estão estruturadas as publicações criadas por usuários no Twitter e como é feita a aquisição desses dados através das interfaces de aplicação, são explicados detalhes do processo de conversão do texto da publicação para o modelo de espaço vetorial, além de como esses dados são posteriormente agrupados.

No Capítulo 4 é apresentado o processo de classificação das publicações através do SVM: o SVM como uma máquina de aprendizado supervisionado, seu modelo de otimização matemático e o que é um SVM de margem suave, assim como o processo de representação de conhecimento possuindo treino e teste e a fase de classificação em si.

No Capítulo 5 demonstra-se como foi feita a implementação do modelo, quais tecnologias foram utilizadas, assim como os resultados obtidos e o ambiente virtual criado para analisar os eventos.

No Capítulo 6 são encontradas a conclusão do trabalho, bem como dificuldades encontradas e trabalhos futuros propostos.